% In this file you should put the actual content of the blueprint.
% It will be used both by the web and the print version.
% It should *not* include the \begin{document}
%
% If you want to split the blueprint content into several files then
% the current file can be a simple sequence of \input. Otherwise It
% can start with a \section or \chapter for instance.

\chapter{Random regular graphs}

In this chapter, we introduce the definitions related to random regular graphs and their spectrum.
We refer the reader to \cite{chung1992} for a more detailed introduction and motivations.

\section{Definition of a graph}

In this section, we define a few simple notions of graph theory, to set conventions.

\begin{definition}
  \lean{Graph}
  \leanok
  A \emph{graph} is a pair $G = (V, E)$ where $V$ is a set of vertices, and $E$ a set of edges,
  and each edge links two unordered, possibly identical vertices.
\end{definition}

Note that this definition allows for loops (i.e. edges going from one vertex to itself) and
multi-edges (i.e. multiple edges connecting two vertices). This convention is essential to the proof
we are aiming for.

\begin{definition}
  \lean{Inc}
  \leanok
  We say an edge $e$ is \emph{indicent} to a vertex $x$ if there exists a vertex $y$ such that $e$
  links $x$ and $y$.
\end{definition}

\begin{definition}
  \lean{IsLoopAt}
  \leanok
  An edge is a \emph{loop} based at a vertex $x$ if it links the vertex $x$ to itself.
\end{definition}

\section{Degree and regular graphs}

Let us now focus on locally finite graphs.

\begin{definition}
  A graph is said to be \emph{locally finite} if any vertex has a finite number of incident edges.
\end{definition}

\begin{definition}
  The \emph{degree} of a vertex $x \in V$ is the number of incident edges, counting loops twice:
  \begin{equation*}
    \mathrm{deg}(x) = \# \{\text{edges incident to } x\}
    + \# \{ \text{loops based at } x\}.
  \end{equation*}
\end{definition}

\begin{definition}
  Let $d$ be a non-negative integer. We say $G$ is $d$-regular if every vertex of $G$ has degree $d$.
\end{definition}

\section{Random regular graphs: the permutation model}

In the rest of this blueprint, $N$ is a positive integer, and $[N] = \{1, \ldots, N\}$.
We also fix an even positive integer $d = 2r$.

\begin{definition}
  We denote as $\mathbf{S}_N$ the set of permutations of $[N]$, i.e. bijective maps $\sigma
  : [N] \rightarrow [N]$.
\end{definition}

\begin{lemma}
  The set $\mathbf{S}_N$ is finite and has cardinality $N!$.
\end{lemma}
We can therefore equip $\mathbf{S}_N$ with the uniform probability measure.

It will be more useful in what follows to view permutations as matrices.

\begin{definition}
  A square matrix is called a \emph{permutation matrix} if all of its entries are equal to $0$ or
  $1$, with exactly one $1$ on each row and column.
\end{definition}

\begin{lemma}
  Let $U$ be a permutation matrix. Then, $U$ is invertible and its inverse is the transpose $U^*$ of $U$.
\end{lemma}

\begin{definition}
  Let $\sigma \in \mathbf{S}_N$. We associate to $\sigma$ the $N \times N$ matrix
  $U_\sigma$ by letting, for each $i, j \in [N]$, $(U_\sigma)_{ij} = 1$ if $j = \sigma(i)$. [todo: check]
\end{definition}

\begin{lemma}
  The map $\sigma \mapsto U_\sigma$ is a bijection from $\mathbf{S}_N$ to the set of $N \times N$
  permutation matrices.
\end{lemma}

\section{The infinite $d$-regular tree}

\begin{definition}
  A \emph{walk} from $x$ to $y$ is a sequence of vertices and edges
  $x = v_0, e_1, v_1, \ldots, e_k, v_k = y$ such that for every $1 \leq j \leq k$, the edge $e_j$
  links the vertices $v_{j-1}$ and $v_j$. A walk is a \emph{path} if its edges are all distinct.
\end{definition}

\begin{definition}
  A graph $G$ is called a \emph{tree} if, for any pair of distinct vertices $x$, $y$, there exists
  one unique path from $x$ to $y$.
\end{definition}

[todo: define the infinite $d$-regular tree]


\chapter{Analytic tools}

Here we define some analytic prerequisites, following the structure of \cite[Section 4]{chen2025}.

\section{Spaces of functions and norms}

[todo: define $C^m[a,b]$ and the norm on it.]

\section{Markov inequalities}

\section{Chebyshev polynomials}

\section{Compactly supported distributions}

[todo: define a distribution and the notion of compactly supported distribution]
