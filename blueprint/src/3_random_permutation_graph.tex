\chapter{Random permutation graphs}


\section{Permutation graphs}

Let $V$ be a finite set and $\perm{V} = \{ \pi : V \rightarrow V \text{ bijective}\}$
the set of permutations of $V$.

We define a graph associated to a family of permutations.

\begin{definition}
  \label{def:dart_of_pi}
  \uses{def:dartlike}
  Let $r \geq 1$ be an integer. To a family $\vec{\pi} = (\pi_1, \ldots, \pi_r) \in \perm{V}^r$
  we associate a dart-like structure on $V$ by taking:
  \begin{itemize}
    \item $D = V \times \{1, \ldots, r\} \times \{-1,+1\}$
    \item for $(x,i,\epsilon) \in D$, we let $\spt(x,i,\epsilon) = x$,
    $\ept(x,i,\epsilon) = \pi_i^\epsilon(x)$ and
    $\symm(x,i,\epsilon) = (\pi_i^\epsilon(x),i,-\epsilon)$.
  \end{itemize}
\end{definition}

\begin{proof}
  We check that
  \begin{equation*}
    \spt(\symm(x,i,\epsilon))
    = \spt(\pi_i^\epsilon(x),i,-\epsilon)
    = \pi_i^\epsilon(x)
    = \ept(x,i,\epsilon)
  \end{equation*}
  and
  \begin{equation*}
    \ept(\symm(x,i,\epsilon))
    = \ept(\pi_i^\epsilon(x),i,-\epsilon)
    = \pi_i^{-\epsilon}(\pi_i^\epsilon(x))
    = x = \spt(x,i,\epsilon)
  \end{equation*}
  and
  \begin{equation*}
    \symm(\symm(x,i,\epsilon))
    = \symm((\pi_i^\epsilon(x),i,-\epsilon))
    = (\pi_i^{-\epsilon}(\pi_i^\epsilon(x)),i,--\epsilon)
    = (x,i,\epsilon).
  \end{equation*}
\end{proof}

\begin{definition}
  \label{def:graph_of_pi}
  \uses{def:dart_of_pi,def:graph_of_dart}
   Let $r \geq 1$ be an integer. To a family $\vec{\pi} \in \perm{V}^r$
  we associate a graph $G_{\vec{\pi}}$ on $V$ by taring the graph associated to the dart-like
  structure associated to $\vec{\pi}$.
\end{definition}

\begin{lemma}
  \label{lem:dartset_pair_pi}
  \uses{def:dartset_pair,def:graph_of_pi}
  Let $r \geq 1$ be an integer and take $\vec{\pi} = (\pi_1, \ldots, \pi_r) \in \perm{V}^r$.
  Then, for any $x, y \in V$,
  \begin{equation*}
    \dartset(x,y) = \{(x,i,\epsilon): y = \pi_i^\epsilon(x)\}.
  \end{equation*}
\end{lemma}

\begin{lemma}
  \label{lem:dartset_pi}
  \uses{lem:dartset_pair_pi}
  Let $r \geq 1$ be an integer and take $\vec{\pi} = (\pi_1, \ldots, \pi_r) \in \perm{V}^r$.
  Then, for any $x \in V$,
  \begin{equation*}
    \dartset(x) = \{(x,i,\epsilon): 1 \leq i \leq r, \epsilon = \pm 1\}.
  \end{equation*}
\end{lemma}

\begin{lemma}
  \label{lem:graph_pi_regular}
  \uses{lem:dartset_pi,def:regular}
  The graph $G_{\vec{\pi}}$ is regular of degree $2r$.
\end{lemma}

[I need to decide the conventions about the side of the actions below]

\begin{definition}
  \label{def:u_pi}
  For $\pi \in \perm{V}$, we define an operator $u_\pi : V^{\C} \rightarrow V^{\C}$ by
  letting, for $f : V \rightarrow \C$,
  \begin{equation*}
    \forall x \in V,
    u_\pi f(x) = f(\pi^{-1}(x))
  \end{equation*}
\end{definition}

\begin{lemma}
  \label{lem:u_pi_unitary}
  \uses{def:u_pi}
  For $\pi \in \perm{V}$, the operator $u_\pi$ is a unitary operator on $\ell^2(V)$
  and $u_\pi^*=u_\pi^{-1}=u_{\pi^{-1}}$.
\end{lemma}

\begin{lemma}
  \label{lem:adjacency_of_pi}
  \uses{def:u_pi,def:adjacency_operator,lem:u_pi_unitary}
  Let $r \geq 1$ be an integer. For any family $\vec{\pi} = (\pi_1, \ldots, \pi_r) \in \perm{V}^r$,
  the adjacency operator $a_{\vec{\pi}}$ of $G_{\vec{\pi}}$ can be written as
  \begin{equation*}
    a_{\vec{\pi}} = \sum_{i=1}^r (u_{\pi_i} + u_{\pi_i}^*).
  \end{equation*}
\end{lemma}

\begin{proof}
  We have by definition of the adjacency operator for a function $f : V \rightarrow \C$
  and $x \in V$,
  \begin{equation*}
    af(x)
    = \sum_{d \in \dartset(x)} f(\ept(d))
    = \sum_{i=1}^r \sum_{\epsilon = \pm 1} f(\pi_i^\epsilon(x))
    =  \sum_{i=1}^r (f(\pi_i^{-1}(x)) + f(\pi_i(x)))
  \end{equation*}
  which leads to the conclusion as we recognise $u_{\pi_i}$ and $u_{\pi_i^{-1}} = u_{\pi_i}^*$.
\end{proof}

\section{Random permutations}

Let $r, N \geq 1$ be integers.
In what follows, we write $\perm{N}$ for the set of permutations of $[N]$
and $\perm{N}^r$ the product space of $r$-tuples of elements of $\perm{N}$.

\begin{definition}
  \label{def:probability_measure}
  We equip $\perm{N}^r$ with the uniform probability measure which we denote as
  $\Prob$. We denote the corresponding expectation by $\expec$.
\end{definition}

\begin{lemma}
  \label{lem:formula_expec}
  \uses{def:probability_measure}
  For any function $F : \perm{N}^r \rightarrow \C$,
  \begin{equation*}
    \expec [F(\vec{\pi})] = \frac{1}{N!^r} \sum_{\vec{\pi} \in \perm{N}^r} F(\vec{\pi}).
  \end{equation*}
\end{lemma}

\begin{proof}
  This is the expectation for a uniform measure, noting that the cardinal of $\perm{N}$
  is $N!^r$.
\end{proof}

\begin{definition}
  \label{def:proba_on_graphs}
  \uses{def:probability_measure,def:graph_of_pi}
  Denote $\de = 2r$.
  Let $\mathcal{G}_{\de,N}$ be the set of $\de$-regular graphs with $V(G) = [N]$,
  equiped with the power-set $\sigma$-algebra.
  We define a probability measure $\ProbG$ on $\mathcal{G}_{\de,N}$ by pushing forward $\Prob$
  with the function
  $\perm{N}^r \rightarrow \mathcal{G}_{\de,N}$ defined in Definition \ref{def:graph_of_pi}.
\end{definition}

\section{Words in random permutation matrices}

The aim of this section is to prove \cite[Lemma 5.1]{chen2025}, a formula for the expectation
of traces of words in random permutation matrices.

\subsection{Statement of the result and notations}

We fix an integer $r \geq 1$.

\begin{definition}
  \label{def:numbering_convention}
  Let $G$ be a group with neutral element $\mathbf{1}$.
  To a family $(x_1, \ldots, x_r) \in G^r$ we associate a family $(x_0, \ldots, x_{2r}) \in G^{2r+1}$
  by letting $x_0= \mathbf{1}$ and for $1 \leq i \leq r$, $x_{r+i} := x_i^{-1}$.
\end{definition}

\begin{definition}
  \label{def:word}
  We denote as $\word$ the set of finite words $w = w_1 \ldots w_q$
  in the letters $\{0, \ldots 2r\}$.
  The \emph{length} $|w| = q$ of a word $w$ is its number of letters.
\end{definition}

\begin{definition}
  \label{def:word_map}
  \uses{def:word,def:numbering_convention}
  Let $w = w_1 \ldots w_q \in \word$
  and $G$ be a group with neutral element $\mathbf{1}$.
  The \emph{word map} $w : G^r \rightarrow G$ is defined by
  letting, for each $(x_1, \ldots, x_r) \in G^r$,
  $w(x_1, \ldots, x_r) = x_{w_1} \ldots x_{w_q}$
  with the numbering convention from Definition \ref{def:numbering_convention}.
\end{definition}

Let $N \geq 1$.
In what follows, we denote as $\matrices$ the set of $N \times N$ matrices with complex coefficients.
If $M \in \matrices$, we denote as $M_{i,j}$ its coefficient $(i,j)$.
We denote as $\TrN M = \sum_{i=1}^N M_{i,i}$ the usual trace on $\matrices$.

\begin{definition}
  \label{def:normalised_trace}
  The \emph{normalised trace} $\trN : \matrices \rightarrow \C$ is defined by $\trN := \frac{1}{N} \TrN$.
\end{definition}

\begin{definition}
  \label{def:gq}
  Let $q \geq 1$ be an integer.
  For $1 \leq j \leq q-1$, we define $r_{r,q,j} := \min (r, \lfloor q/(j+1) \rfloor)$
  and
  we define the polynomial $g_{r,q}(x) = \prod_{j=1}^{q-1} (1-j x)^{r_{r,q,j}}$.
\end{definition}

\begin{lemma}
  \label{lem:degree_gq}
  \uses{def:gq}
  For any $q \geq 1$, the degree of $g_{r,q}$ is at most $q(1+\log r)-1$.
\end{lemma}

\begin{proof}
  By comparaison with an integral, we have
  \begin{align*}
    \deg g_{r,q}
    = \sum_{j=1}^{q-1} r_{r,q,j}
    \leq \int_1^q \min(r, q/x) \d x
    \leq q(1+\log r) - \min(r,q).
  \end{align*}
\end{proof}

We are now ready to state the main result of this section.

\begin{theorem}
  \label{thm:expectation_word_formula}
  \uses{def:word_map,def:gq,def:normalised_trace}
  \notready
  Let $q \geq 1$.
  Let $w \in \word$ be a word of length at most $q$.
  There exists a polynomial $f_w$, of degree at most $q(1+\log r)$, such that for any $N \geq q$,
  \begin{equation*}
    \expec [\trN w(\vec \pi)] = \frac{f_w(1/N)}{g_q(1/N)}.
  \end{equation*}
\end{theorem}
