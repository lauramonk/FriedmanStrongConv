\chapter{Spectral theory of regular graphs}

In this chapter, we introduce the definitions related to regular graphs and their spectrum.
We refer the reader to \cite{chung1992} for a more detailed introduction and motivations.

As in Graph, a \emph{graph} is a pair $G = (V, E)$ where $V$ is a set of vertices, and
$E$ a set of edges, and each edge links two unordered, possibly identical vertices.
The notations $V(G)$ and $E(G)$ respectively denote the subsets of $V$ and $E$ corresponding to
vertices or edges of $G$.

Note that this definition allows for loops (i.e. edges going from one vertex to itself) and
multi-edges (i.e. multiple edges connecting two vertices). This convention is essential to the proof
we are aiming for, which is why we use Graph instead of the more developed
SimpleGraph.

Following Inc, we say an edge $e$ is \emph{incident} to a vertex $x$ if there exists a
vertex $y$ such that $e$ links $x$ and $y$. As IsLoopAt, an edge is a \emph{loop} based
at a vertex $x$ if it links the vertex $x$ to itself.

\section{Basic notions of graph theory}

Let $G$ be a graph.
Let us define a few basic notions from graph theory.
We make definitions with minimal
hypotheses as these are fundamental objects that should be added to Mathlib.

\subsection{General definitions}

\begin{definition}
\label{def:empty}
We say $G$ is empty if $V(G) = \emptyset$.
\end{definition}

The definition of subgraph is easier for Graph as the vertexset is just a subset.

\begin{definition}
  \label{def:subgraph}
  A \emph{subgraph} $H$ of $G$ is a graph of type $\alpha$ $\beta$ defined by:
  \begin{itemize}
    \item a vertexset included in the vextex set of $G$
    \item an edgeset included in the edge set of $H$
    \item the same $\islink$ relation.
  \end{itemize}
\end{definition}

\begin{lemma}
  \label{lem:subgraph_adj_graph}
  \uses{def:subgraph}
  Let $H$ be a subgraph of $G$.
  For any $x$, $y$ in $V(H)$, if $x$ and $y$ are adjacent for $H$, then they are
  adjacent for $G$.
\end{lemma}

\begin{definition}
  \label{def:max_subgraph,def:min_subgraph}
  We define the \emph{union}, resp. \emph{intersection} of two subgraphs by taking the union of
  their respective vertexsets and edgesets.
\end{definition}

\begin{definition}
  \label{def:isinduced}
  A subgraph $H$ of $G$ is \emph{induced} if, for any vertex of $H$, any edge of $G$ incident
  to $x$ is an edge of $H$.
\end{definition}

We will be interested in the Hilbert space $\ell^2(G) := \ell^2(V(G))$, which is defined as the set of
functions $f : V(G) \rightarrow \C$ which are square-summable, equipped with the inner product
$$\langle f, g \rangle = \sum_{x \in V(G)} f(x) \overline{g(x)}.$$

The space $\B(\ell^2(G))$ of bounded linear operators on $\ell^2(G)$ is a unital $\C^*$-algebra.
Note that, in this context, $0$ is the constant function equal to $0$.

\subsection{Walks}

We define a notion of walk on the graph.

\begin{definition}
  \label{def:walk,def:length}
  \leanok
  Let $x, y \in V(G)$. A \emph{walk} from $x$ to $y$ on $G$ is given by two finite lists:
  \begin{itemize}
    \item a list $(v_i)_{0 \leq i \leq m}$ of vertices in $V(G)$;
    \item a list $(e_i)_{1 \leq i \leq m}$ of edges in $E(G)$;
  \end{itemize}
  satisfying the following:
    \begin{itemize}
      \item $v_0=x$ and $v_m=y$;
      \item for any $i \in \{1, \ldots, m\}$, $e_i$ links $v_{i-1}$ and $v_i$.
    \end{itemize}
  The integer $m$ is the \emph{length} of the walk.
\end{definition}

\begin{definition}
  \label{def:reachable}
  \uses{def:walk}
  \leanok
  We define a binary relation on $V(G)$ by: $x \sim y$ if there exists a walk
  from $x$ to $y$ on $G$. We then say $x$ and $y$ are \emph{reachable}.
\end{definition}

This is an equivalence relation.

\begin{proposition}
  \label{prop:reachable_is_equivalence,lem:reachable_reflexive,lem:reachable_symmetric,
  lem:reachable_transitive}
  The reachability relation is an equivalence relation on $V(G)$.
\end{proposition}

\begin{proof}
  The reachability relation is reflexive: $x \sim x$ for any $x \in V(G)$.
  Indeed, let $x \in V(G)$. We take $m=0$ and set $v_0=x$. Then this is a walk from
  $x$ to $x$ on $G$.

  The reachability relation is symmetric: if $x \sim y$ then  $y \sim x$.
  Let $x$, $y$ be two vertices. We assume $x \sim y$. Let $(v_i)_{0 \leq i \leq m}$ and $(e_i)_{1 \leq i \leq m}$
  be a walk from $x$ to $y$. Then $(v_{m-i})_{0 \leq i \leq m}$ and $(e_{m+1-i})_{1 \leq i \leq m}$
  is a walk from $y$ to $x$ by the is\_link\_symm. So $y \sim x$.

  The reachability relation is transitive: if $x \sim y$ and $y \sim z$ then
  $x \sim z$.
  Assume $x \sim y$ and $y \sim z$. Let $(v_i)_{0 \leq i \leq m}$ and $(e_i)_{1 \leq i \leq m}$
  be a walk from $x$ to $y$ and $(v_i')_{0 \leq i \leq m'}$, $(e_i')_{1 \leq i \leq m'}$
  be a walk from $y$ to $z$.
  We let $m''=m+m'$ and define for $i \in \{0, \ldots, m'\}$
  \begin{equation}
    v_i'' := \begin{cases}
      v_i & \text{if } i \leq m \\
      v_{i-m-1}' & \text{otherwise}
    \end{cases}
  \end{equation}
 and for $i \in \{1, \ldots, m'\}$
   \begin{equation}
    e_i'' := \begin{cases}
      e_i & \text{if } i \leq m \\
      e_{i-m}' & \text{otherwise}.
    \end{cases}
  \end{equation}
  Then this is a walk from $x$ to $z$, hence $x \sim z$.
\end{proof}

The following will be useful.

\begin{lemma}
  \label{lem:adj_towalk}
  \uses{def:reachable}
  Let $x$, $y$ in $V(G)$. If $x$ and $y$ are adjacent then $x$ and $y$ are reachable.
\end{lemma}

\begin{proof}
  By definition of adjacency, there exists an edge $e \in E(G)$ linking $x$ and $y$.
  Then the walk $(x, y)$ and $(e)$ is a walk from $x$ to $y$.
\end{proof}

\subsection{Connectedness}

We define the notion of connectedness and the connected components of a graph.

\begin{definition}
  \label{def:preconnected}
  \uses{def:reachable}
  A graph is \emph{preconnected} if any pair of vertices is reachable.
\end{definition}

\begin{definition}
  \label{def:connected}
    \uses{def:preconnected,def:empty}
  The graph $G$ is \emph{connected} if it is preconnected and not empty.
\end{definition}

\begin{lemma}
  \label{lem:connected_iff_exists_forall_reachable}
  The graph $G$ is connected if and only if there exists $x \in V(G)$ such that,
  for all $y \in V(G)$, $x$ and $y$ are reachable.
\end{lemma}

\begin{definition}
  \label{def:connected_component}
  \uses{def:connected}
  A \emph{connected component} of $G$ is an equivalence class of $V(G)$ for
  the relation reachable.
\end{definition}

\begin{definition}
  \label{def:connected_component_Mk}
  \uses{def:connected_component}
  For any vertex $x$, we define $\ccm(x)$ to be the unique connected component
  of $G$ containing $x$.
\end{definition}

\begin{theorem}
  \label{thm:cc_eq}
  \uses{def:connected_component_Mk}
  Let $x, y \in V(G)$. Then $\ccm(x) = \ccm(y)$ iff $x$ and $y$ are reachable.
\end{theorem}

\begin{definition}
  \label{def:subgraph_cc}
  \uses{def:connected_component,def:subgraph}
  Let $C$ be a  connected component of $G$.
  We define a subgraph associated to $C$ by taking $C$ as the vertexset,
  and $\{e \in E(G) \, : \, e \text{ incident to a } x \in C\}$ as edgeset.
\end{definition}

\begin{lemma}
  \label{lem:subgraph_cc_is_induced}
  \uses{def:isinduced,def:subgraph_cc}
  The subgraph associated to a connected component is induced.
\end{lemma}

\begin{theorem}
  \label{thm:cc_is_connected}
  \uses{def:subgraph_cc,def:connected}
  Let $C$ be a connected component of $G$. Then the associated subgraph is connected.
\end{theorem}

\begin{definition}
  \label{def:set_cc}
  \uses{def:connected_component}
  We denote as $\cc(G)$ the set of connected components of $G$.
\end{definition}

\begin{proposition}
  \label{prop:cc_is_partition}
  \uses{def:set_cc}
  $\cc(G)$ is a partition of $V(G)$.
\end{proposition}

\begin{proof}
  This comes from the description as an equivalence relation.
\end{proof}

\subsection{Notions of finiteness}

\begin{definition}
  \label{def:locally_finite_at}
  Let $x$ be a vertex of $G$.
  We say $G$ is \emph{locally finite at $x$} if $\incset (x)$ is finite.
\end{definition}

\begin{definition}
  \label{def:locally_finite}
  \uses{def:locally_finite_at}
  $G$ is said to be \emph{locally finite} if, for all $x \in V(G)$, $G$ is locally finite at $x$.
\end{definition}

\begin{definition}
  \label{def:finite}
  We say $G$ is \emph{finite} if $V(G)$ and $E(G)$ are both finite.
\end{definition}

\begin{lemma}
  \label{lem:all_l2_finite}
  Assume $V(G)$ is finite. Then any function $V(G) \rightarrow \C$ is in $\ell^2(G)$.
\end{lemma}

\begin{lemma}
  \label{lem:bound_number_cc}
  \uses{prop:cc_is_partition}
  If $V(G)$ is finite, then its number of connected components is finite and at most $\# V(G)$.
\end{lemma}

\begin{proof}
  This follows from Proposition \ref{prop:cc_is_partition}
  and the fact that the number of elements in a partition of a finite set is at most its cardinal.
\end{proof}

% \begin{proposition}
%   \label{def:finite_vs_locally_finite}
%   \uses{def:finite,def:locally_finite}
%   $G$ is finite if and only if $V(G)$ is finite and $G$ is locally finite.
% \end{proposition}

\subsection{Degree and regularity}

Not ready as will be refactored using darts!

\begin{definition}
  \label{def:degree}
  \uses{def:locally_finite_at}
  Let $x$ be a vertex of $G$. We assume $G$ is locally finite at $x$.
  The \emph{degree} of $x$ is the number of incident edges, counting loops twice:
  \begin{equation*}
    \mathrm{deg}(x) = \sum_{e \in \incset(x)}
    2^{\isloop(e)}.
  \end{equation*}
\end{definition}

\begin{definition}
  \label{def:regular}
  \uses{def:locally_finite, def:degree}
  Let $d$ be a non-negative integer.
  We say $G$ is \emph{regular of degree $d$} if it is not empty, locally finite and,
  for every $x \in V(G)$, $\deg (x) = d$.
\end{definition}

\begin{definition}
  \label{def:locally_bounded}
  \uses{def:degree,def:locally_finite}
  We say $G$ is \emph{locally bounded} if
  it is locally finite and there exists a constant $D$ such that,
  for all $x \in V(G)$, $\deg(x) \leq D$.
\end{definition}

\subsection{Constant fonctions}

\begin{definition}
  \label{def:constant_function}
  A function $V(G) \rightarrow \C$ is \emph{constant} if, for all $x, y \in V(G)$,
  $f(x)=f(y)$.
\end{definition}

\begin{definition}
  \label{def:constant_one}
  The constant function equal to $1$ is defined by
  \begin{equation}
    \const :
    \begin{cases}
      V(G) & \rightarrow \C \\
      x & \mapsto 1.
    \end{cases}
  \end{equation}
\end{definition}

\begin{lemma}
  \label{lem:constant_neq_zero}
  \uses{def:empty,def:constant_function}
  Assume $G$ is not empty. Then the constant function $\const$ is not equal to $0$.
\end{lemma}

\begin{proof}
  If $G$ is not empty, then by Definition \ref{def:empty}, there exists $x \in V(G)$.
  Then $\const (x) = 1 \neq 0$, and hence $\const$ is not equal to the constant function $0$.
\end{proof}

\begin{theorem}
  \label{thm:space_constant}
  Assume $G$ is not empty. Then the space of constant functions is
  a vectorial space of dimension $1$, generated by $\const$.
\end{theorem}

\begin{proof}
  We directly check that:
  \begin{itemize}
    \item the constant function equal to $0$ is locally constant;
    \item for any locally constant functions $f_1$, $f_2$ and any $\lambda \in \C$,
    the function $f_1+cf_2$ is also locally constant;
  \end{itemize}
  both by direct application of the definition.
  This space contains the function $\const$.
  Let $f : V(G) \rightarrow \C$ be a constant function. Since $V(G)$ is not
  empty there exists $x \in V(G)$. Then for all $y \in V(G), f(y) = f(x) \const(y)$,
  so $f = f(x) \const$, which allows to conclude.
\end{proof}

\begin{lemma}
  \label{lem:constant_is_l2}
  \uses{def:constant_function}
  The space of constant functions is included in $\ell^2(G)$
  if and only if $V(G)$ is finite.
\end{lemma}

\begin{proof}
  This is simply the fact that
  non-zero constant sequences are summable if and only if the set is finite.
\end{proof}

\subsection{Locally constant functions}

\begin{definition}
  \label{def:locally_constant}
  A function $f : V(G) \rightarrow \C$ is said to be \emph{locally constant}
  if, for all $x,y \in V(G)$, if $x$ is adjacent to $y$, then $f(x)=f(y)$.
\end{definition}

\begin{lemma}
  \label{lem:locally_constant_vector_space}
  \uses{def:locally_constant}
  The space of locally constant functions is a vectorial subspace of
  the space of functions $V(G) \rightarrow \C$.
\end{lemma}

\begin{proof}
  We directly check that:
  \begin{itemize}
    \item the constant function equal to $0$ is locally constant;
    \item for any locally constant functions $f_1$, $f_2$ and any $\lambda \in \C$,
    the function $f_1+cf_2$ is also locally constant;
  \end{itemize}
  both by direct application of the definition.
\end{proof}

\begin{lemma}
  \label{lem:locally_constant_walk}
  \uses{def:reachable,def:locally_constant}
  Let $f : V(G) \rightarrow \C$ be a locally constant function.
  For any $x, y \in V(G)$, if $x$ and $y$ are reachable, then $f(x)=f(y)$.
\end{lemma}

\begin{proof}
  By induction.
\end{proof}

\begin{lemma}
  \label{lem:locally_constant_preconnected}
  \uses{lem:locally_constant_walk,def:preconnected,def:constant_function}
  A locally constant function on a preconnected graph is constant.
\end{lemma}

\begin{proof}
  Let $G$ be a preconnected graph and $f : V(G) \rightarrow \C$ be a locally
  constant function.
  For any $x$, $y$ in $V(G)$, by Definition \ref{def:preconnected}, there
  exists a walk from $x$ to $y$. Lemma \ref{lem:locally_constant_walk}
  this implies $f(x)=f(y)$. Hence by Definition \ref{def:constant_function},
  $f$ is constant.
\end{proof}

\begin{lemma}
  \label{lem:locally_constant_subgraph}
  \uses{lem:subgraph_adj_graph,def:locally_constant}
  The restriction of a locally constant function
  to a subgraph is locally constant on that subgraph.
\end{lemma}

\begin{proof}
  Let $f$ be a locally constant function on $G$, and $H$ be a subgraph of $G$.
  Let $x, y$ be adjacent vertices in $H$. By Lemma \ref{lem:subgraph_adj_graph},
  $x$ and $y$ are also adjacent in $G$. Then $f(x)=f(y)$.
\end{proof}

\begin{lemma}
  \label{lem:locally_constant_cc}
  \uses{lem:locally_constant_subgraph,thm:cc_is_connected,
  lem:locally_constant_preconnected,thm:cc_eq}
  A function $f$ is locally constant if and only if its restriction to each
  connected component is a constant function.
\end{lemma}

\begin{proof}
  Assume $f$ is locally constant. Let $C$ be a connected component of $G$.
  By Lemma \ref{lem:locally_constant_subgraph}, the restriction of $f$ to $C$
  is locally constant.
  By Theorem \ref{thm:cc_is_connected}, $C$ is preconnected.
  So by Lemma \ref{lem:locally_constant_preconnected}, it is constant.

  Assume the restriction of $f$ to each connected component is constant.
  Let $x$, $y$ be adjacent vertices. Then by Theorem \ref{thm:cc_eq},
  $\ccm(x) = \ccm(y)$ which allows to conclude since $f$ is constant on $\ccm(x)$
  by hypothesis and Theorem \ref{thm:cc_is_connected}.
\end{proof}

\begin{definition}
  \label{def:indicator_function}
  Let $U$ be a subset of $V(G)$. We define the \emph{indicator function}
  $\const_U : V(G) \rightarrow \C$ by
  \begin{equation}
    \const_U (x) =
    \begin{cases}
      1 & \text{if } x \in U \\
      0 & \text{otherwise}.
    \end{cases}
  \end{equation}
\end{definition}

\begin{lemma}
  \label{lem:indicator_cc_loc_const}
  \uses{def:locally_constant,def:indicator_function}
  The indicator function of a connected component is locally constant.
\end{lemma}

\begin{lemma}
  \label{lem:partition_unity_cc}
  \uses{prop:cc_is_partition,def:indicator_function}
  We have $$\const = \sum_{C \in \cc(G)} \const_C.$$
\end{lemma}

\begin{proof}
  Let $x \in V(G)$.
  By Lemma \ref{lem:partition_unity_cc}, there exists a unique
  connected component of $G$ containing $x$.
  Hence, $\sum_{C \in \cc(G)} \const_C(x)=1$.
\end{proof}

\begin{proposition}
  \label{prop:basis_locally_constant}
  \uses{lem:locally_constant_cc,lem:partition_unity_cc}
  The family $\{\const_C, C \in \cc(G)\}$ is a basis
  of the space of locally constant functions.
\end{proposition}

\begin{proof}
  First,  for any connected component $C$, $\const_C$ is locally
  constant by Lemma \ref{lem:indicator_cc_loc_const}.

  Let $f$ be a locally constant function.
  We have by Lemma \ref{lem:partition_unity_cc}, for any $x \in V(G)$,
  \begin{equation*}
    f(x) = \sum_{C \in \cc(G)} f(x) \const_C(x).
  \end{equation*}
  By Lemma \ref{lem:locally_constant_cc}, for every connected component $C$,
  there exists a constant $c_C$ such that, for all $x \in C$, $f(x)=c_C$
  (take $c_C$ to be $f(y_C)$ for a $y_C \in C$, which exists as $C$ is not empty).
  Then
    \begin{equation*}
    f(x) = \sum_{C \in \cc(G)} c_C \const_C(x)
  \end{equation*}
  which means that $f = \sum_{C \in \cc(G)} c_C \const_C$.
\end{proof}

\begin{corollary}
  \label{cor:dimension_locally_constant}
  \uses{prop:basis_locally_constant}
  The dimension of the space of locally constant functions is the number of
  connected components of $G$.
\end{corollary}

\subsection{Colorings and bipartiteness}

\begin{definition}
  \label{def:bipartite}
  The graph $G$ is \emph{bipartite} if there exists a partition $U_1$,
  $U_2$ of $V(G)$ such that, for all $e \in E(G)$, $e$ links an element of $U_1$
  and an element of $U_2$.
\end{definition}

\section{Adjacency operator}

Let $G$ be a graph.

\subsection{Definition of the adjacency operator and action on constants}

Assume $G$ is locally finite.

\begin{definition}
  \label{def:adjacency_operator}
  \uses{def:locally_finite}
  We define the adjacency operator $a$ as the linear operator acting on the space of
  functions $f : V(G) \rightarrow \C$ by
  $$ a f (x) :=
  \sum_{e \in \incset(x)}
  2^{\isloop(e)} f(\incother(e,x)).$$
\end{definition}

\begin{lemma}
  \label{lem:adjacency_of_constant}
  \uses{def:constant_function,def:adjacency_operator}
  For any $x \in V(G)$, $a \const(x) = \deg(x)$.
\end{lemma}

\begin{proof}
  Let $x \in V(G)$. We have
  \begin{align*}
    a \const(x)
    & = \sum_{e \in \incset(x)}
    2^{\isloop(e)}\const(\incother(e,x))
    & \text{by Definition \ref{def:adjacency_operator}} \\
    & = \sum_{e \in \incset(x)}
    2^{\isloop(e)}
    & \text{by Definition \ref{def:constant_function}} \\
    & = \deg(x)
    & \text{by Definition \ref{def:degree}}.
  \end{align*}
\end{proof}

\subsection{Norm of the adjacency operator}

Let $G$ be a locally finite graph.
The aim of this subsection is to prove the following.

\begin{proposition}
  \label{prop:norm_adjacency_leq_sup_degree}
  \uses{lem:bound_a_sup_degree}
  Assume $G$ is locally bounded.
  Then the adjacency operator is a bounded operator on $\ell^2(G)$ and
  its norm satisfies $\|a\| \leq \sup_{x \in V(G)} \deg(x)$.
\end{proposition}

Here is a key lemma.

\begin{lemma}
  \label{lem:bound_a_degree_x}
  \uses{def:adjacency_operator,def:degree}
  For any function $f : V(G) \rightarrow \C$, any vertex $x \in V(G)$,
  \begin{equation}
    |af(x)|^2
    \leq \deg(x)
    \sum_{e \in \incset(x)} 2^{\isloop(e)} |f(\incother(e,x))|^2.
  \end{equation}
  Furthermore, this is an equality if and only if there exists $c \in \C$ such that
  \begin{equation}
    \forall e \in \incset(x), f(\incother(e,x)) = c.
  \end{equation}
\end{lemma}

\begin{proof}
  Let $f : V(G) \rightarrow \C$ and $x \in V(G)$.
  By Definition \ref{def:adjacency_operator},
  \begin{equation}
    |af(x)|^2
    = \left| \sum_{e \in \incset(x)} 2^{\isloop(e)} f(\incother(e, x))\right|^2.
  \end{equation}
  By the Cauchy-Schwarz inequality applied to the complex vectors
  \begin{equation}
    v_1 := (2^{\isloop(e)/2})_{e \in \incset(x)}
  \end{equation}
  and
  \begin{equation}
    v_2 := (2^{\isloop(e)/2} f(\incother(e,x)))_{e \in \incset(x)}
  \end{equation}
  we have
  \begin{align*}
    |af(x)|^2
    & = \left| v_1 \cdot v_2 \right|^2
    \leq \|v_1\|_2^2 \|v_2\|_2^2 \\
    &\leq
    \left(\sum_{e \in \incset(x)} 2^{\isloop(e)}
    \right)
    \left( \sum_{e \in \incset(x)} 2^{\isloop(e)} |f(\incother(e, x))|^2
    \right).
  \end{align*}
  This leads to the inequality by Definition \ref{def:degree}.
  The equality case comes from the equality case in Cauchy-Schwarz, which occurs if
  and only if the vectors $v_1$ and $v_2$
  are colinear, which
  is equivalent to the announced condition as $v_1$ is a non-zero constant vector.
\end{proof}

Then Proposition \ref{prop:norm_adjacency_leq_sup_degree} is a trivial consequence of the
following lemma, together with the definition of bounded operator and operator norm.

\begin{lemma}
  \label{lem:bound_a_sup_degree}
  \uses{lem:bound_a_degree_x,def:locally_bounded}
  Assume $G$ is locally bounded. Then, for any $f \in \ell^2(G)$,
  $a f \in \ell^2(G)$ and
  \begin{equation}
    \|a f\| \leq \sup_{x \in V(G)} \deg(x) \times \|f\|.
  \end{equation}
  Furthermore, the equality occurs if and only if $G$ is regular
  and, for any $x \in V(G)$, there exists $c \in \C$ such that
    \begin{equation}
    \forall e \in \incset(x), f(\incother(e,x)) = c.
  \end{equation}
\end{lemma}

\begin{proof}
  To write, basically we need to sum Lemma \ref{lem:bound_a_degree_x} for $x \in V(G)$
  (find the correct summation result), change the order of summation (same)
  and observe that
  \begin{align*}
    \sum_{x \in V(G)}
    \sum_{e \in \incset(x)}
    2^{\isloop(e)}
    |f(\incother(e,x))|^2
    & = \sum_{x \in V(G)} \deg(x) |f(x)|^2 \\
    & \leq \sup_{x \in V(G)} \deg(x) \times \|f\|^2.
  \end{align*}
  If the equality occurs, since all terms are non-negative, it means that
  $\deg (x)$ is constant (i.e. the graph is regular), and that there is equality in
  each term of  Lemma \ref{lem:bound_a_degree_x}, which is the conclusion.
\end{proof}

\subsection{Adjacency operator is self-adjoint}

Let $G$ be a locally bounded graph.

\begin{proposition}
  \label{prop:adjacency_is_selfadjoint}
  \uses{lem:scalar_of_adjacency}
  The adjacency operator is selfadjoint.
\end{proposition}

The following lemma will be very handy.

\begin{lemma}
  \label{lem:scalar_of_adjacency}
  \uses{lem:bound_a_sup_degree}
  For any $f$, $g$ in $\ell^2(G)$,
  \begin{equation}
    \langle af, g \rangle
    = \sum_{\substack{e \in E(G)\\ e \text{ links } x,y}}
    (f(x) \overline{g(y)} + f(y) \overline{g(x)}).
  \end{equation}
\end{lemma}

\begin{proof}
  Let $f$, $g$ be elements of $\ell^2(G)$.
  By Proposition \ref{lem:bound_a_sup_degree}, $a f \in \ell^2(G)$.
  By definition of the scalar product,
  \begin{equation}
    \langle af, g \rangle
    = \sum_{x \in V(G)} af(x) \overline{g(x)}.
  \end{equation}
  By Definition \ref{def:adjacency_operator},
  \begin{equation}
    \langle af, g \rangle
    = \sum_{x \in V(G)} \sum_{e \in \incset(x)} 2^{\isloop(e)}
    f(\incother(e,x)) \overline{g(x)}
  \end{equation}
  (todo: justify the change of order of summation, notably using Lemma
  \ref{lem:bound_a_sup_degree} to have absolute convergence)
  which we can reorder to the result.
\end{proof}

We are now ready to prove Proposition \ref{prop:adjacency_is_selfadjoint}.

\begin{proof}
  Let $f$, $g$ be elements of $\ell^2(G)$.
  We have by Lemma \ref{lem:scalar_of_adjacency}
  \begin{equation*}
    \langle af, g \rangle
    = \sum_{\substack{e \in E(G)\\ e \text{ links } x,y}}
    (f(x) \overline{g(y)} + f(y) \overline{g(x)})
    = \overline{\sum_{\substack{e \in E(G)\\ e \text{ links } x,y}}
    (g(x) \overline{f(y)} + g(y) \overline{f(x)})}
  \end{equation*}
  by complex conjugation, commutativity of addition and isLink\_symm.
  Hence by Lemma \ref{lem:scalar_of_adjacency}
  \begin{equation*}
    \langle af, g \rangle
    = \overline{\langle ag, f \rangle} = \langle f, ag \rangle
  \end{equation*}
  by inner\_conj\_symm. This means that $a^*=a$ by definition of
  the adjoint, which means $a$ is self-adjoint by definition.
\end{proof}

\begin{corollary}
  \label{cor:spectrum_is_real}
  \uses{prop:adjacency_is_selfadjoint}
  The spectrum of the adjacency matrix is real.
\end{corollary}

\begin{proof}
  This follows from Proposition \ref{prop:adjacency_is_selfadjoint} with
  selfAdjoint.mem\_spectrum\_eq\_re.
\end{proof}

\begin{lemma}
  \label{lem:spectrum_in_segment}
  \uses{cor:spectrum_is_real,prop:norm_adjacency_leq_sup_degree}
  The spectrum of $a$ is included in the
  segment $[-D,D]$ where $D = \sup_{x \in V(G)} \deg(x)$.
\end{lemma}

\begin{proof}
  By Corollary \ref{cor:spectrum_is_real}, the spectrum of $a$ is real.
  By spectrum.subset\_closedBall\_norm,
  it is included in the closed ball of center $0$ and
  radius $\|a\|$.
  We have $\|a\| \leq \sup_{x \in V(G)} \deg(x)$ by Proposition
   \ref{prop:norm_adjacency_leq_sup_degree}.
\end{proof}

\subsection{The case of regular graphs}

Let $d$ be a non-negative integer.

\begin{lemma}
  \label{lem:adjacency_of_constant_regular}
  \uses{lem:adjacency_of_constant,def:regular}
  Assume $G$ is not empty and locally finite. Then
  $G$ is regular of degree $d$ iff $a \const = d \const$.
\end{lemma}

\begin{proof}
  By Definition \ref{def:regular},
  $G$ is regular of degree $d$ if and only if for all $x \in V(G)$, $\deg(x)=d$.
  By Lemma \ref{lem:adjacency_of_constant}, this is equivalent to
  having, for all $x \in V(G)$, $a \const (x) = d$, which is equivalent to
  $a \const = d \const$ by Definition \ref{def:constant_function}.
\end{proof}

\begin{lemma}
  \label{lem:norm_adjacency_regular_leq_degree}
  \uses{lem:bound_a_sup_degree}
  If $G$ is regular of degree $d$, then $\| a \| \leq d$.
\end{lemma}

\begin{proof}
  This is simply Proposition \ref{prop:norm_adjacency_leq_sup_degree}.
\end{proof}

\begin{lemma}
  \label{lem:spectrum_regular_in_segment}
  \uses{lem:spectrum_in_segment}
  If $G$ is regular of degree $d$, then the spectrum of $a$ is included in the
  segment $[-d,d]$.
\end{lemma}

\begin{proof}
  This is simply Lemma \ref{lem:spectrum_in_segment}.
\end{proof}

\subsection{The case of finite graphs}

\begin{theorem}
  \label{thm:spectral_finite}
  \uses{prop:adjacency_is_selfadjoint}
  Assume $G$ is finite and let $n$ denote the cardinal of $V(G)$.
  Then the spectrum of the adjacency operator is a family of $n$ real eigenvalues
  $$\lambda_1 \geq \lambda_2 \geq \ldots \geq \lambda_n.$$
\end{theorem}

\begin{proof}
  This is the spectral theorem in finite dimension.
\end{proof}

\subsection{The case of finite regular graphs}

Let $d$ be a non-negative integer.

\begin{lemma}
  \label{lem:constant_eigenfunction_regular}
  \uses{lem:constant_is_l2,lem:constant_neq_zero,lem:adjacency_of_constant_regular}
  Assume $G$ is regular of degree $d$ and finite. Then
  $d$ is an eigenvalue of the adjacency
  operator with associated eigenfunction the constant function $\const$.
\end{lemma}

\begin{proof}
  Since $G$ is finite, $\const \in \ell^2(G)$ by Lemma \ref{lem:constant_is_l2}.
  Furthermore, $\const \neq 0$ by Lemma \ref{lem:constant_neq_zero}
  and since $G$ is not empty by Definition \ref{def:regular}.
  By Lemma \ref{lem:adjacency_of_constant_regular}, if $G$ is regular of degree $d$,
  then $a \const = d \const$.  We conclude by the definition of eigenvalue and eigenvector.
\end{proof}

\begin{proposition}
  \label{prop:norm_adjacency_regular}
  \uses{lem:norm_adjacency_regular_leq_degree,lem:constant_eigenfunction_regular}
  If $G$ is finite and regular of degree $d$, then $\|a\|=d$.
\end{proposition}

\begin{proof}
  Since $G$ is regular of degree $d$, by
  Lemma \ref{lem:norm_adjacency_regular_leq_degree}, $\|a \| \leq d$.
  By Lemma \ref{lem:constant_eigenfunction_regular}, $d$ is an eigenvalue of $a$ and hence $\|a\| \geq d$,
  which is enough to conclude.
\end{proof}

\begin{proposition}
  \label{prop:regular_multiplicity_trivial_eigenvalue}
  \uses{def:number_cc,thm:spectral_finite}
  Assume $G$ is finite and regular of degree $d$. Then the multiplicity of the
  eigenvalue $d$ is exactly the number of connected components of $G$.
\end{proposition}

\begin{proof}
  todo
\end{proof}

\begin{proposition}
  \label{prop:eigenvalue_regular_bipartite}
  \uses{def:bipartite,def:indicator_function,lem:all_l2_finite}
  Assume $G$ is finite, regular of degree $d \geq 1$ and connected. Then
  $-d$ is an eigenvalue of $a$ if and only if $G$ is bipartite.
\end{proposition}

\begin{proof}
  Assume $G$ is bipartite. By Definition \ref{def:bipartite}, there exists a partition
  $U_1$, $U_2$ of $V(G)$ such that any edge of $G$ links a vertex of $U_1$ and a
  vertex of $U_2$.
  Let $f := \const_{U_1} - \const_{U_2}$.
  The set $U_1$ is non-empty as there exists a edge in the graph, and $U_1$ contains
  one of its endpoints. Hence $f \neq 0$.
  By Lemma \ref{lem:all_l2_finite}, $f$ is an element of $\ell^2(G)$.
  Let $x \in V(G)$. Let $y$ be adjacent to $x$.
  By definition of $U_1$ and $U_2$,
  $x \in U_1 \Leftrightarrow y \in U_2$. Hence $f(x)=-f(y)$.
  Hence summing over all adjacent vertices we find
  \begin{equation}
    af(x) = \sum_{e \in \incset(x)} 2^{\isloop(e)} f(\incother(e,x))
    = - \sum_{e \in \incset(x)} 2^{\isloop(e)} f(x) = - d f(x).
  \end{equation}
  Hence $af=-df$ which means $-d$ is an eigenvalue of $a$.

  Now assume $-d$ is an eigenvalue of $a$. Take $f$ an eigenvector.
  We have $af = -d f$
  by definition, and hence $\|af\| = d \|f\|$.
  Let $x \in V(G)$.
  We can apply the equality case of Lemma \ref{lem:bound_a_sup_degree}
  and obtain a constant $c \in \C$ such that
  \begin{equation}
    \forall e \in \incset(x), f(\incother) = c.
  \end{equation}
  In particular,
  \begin{equation}
    -d f(x) = af(x) = \sum_{e \in \incset(x)} 2^{\isloop(e)} c = d c.
  \end{equation}
  Since $d \neq 0$, this implies $c=-1$. We have hence proved that
  for any adjacent vertices $x, y$, $f(x)=-f(y)$.
[finish]
\end{proof}
