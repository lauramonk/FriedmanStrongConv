\chapter{Spectral theory of regular graphs}

In this chapter, we introduce the definitions related to regular graphs and their spectrum.
We refer the reader to \cite{chung1992} for a more detailed introduction and motivations.

As in \mathlibok{Graph}, a \emph{graph} is a pair $G = (V, E)$ where $V$ is a set of vertices, and
$E$ a set of edges, and each edge links two unordered, possibly identical vertices.
The notations $V(G)$ and $E(G)$ respectively denote the subsets of $V$ and $E$ corresponding to
vertices or edges of $G$.

Note that this definition allows for loops (i.e. edges going from one vertex to itself) and
multi-edges (i.e. multiple edges connecting two vertices). This convention is essential to the proof
we are aiming for, which is why we use \mathlibok{Graph} instead of the more developed
\mathlibok{SimpleGraph}.

Following \mathlibok{Inc}, we say an edge $e$ is \emph{incident} to a vertex $x$ if there exists a
vertex $y$ such that $e$ links $x$ and $y$. As \mathlibok{IsLoopAt}, an edge is a \emph{loop} based
at a vertex $x$ if it links the vertex $x$ to itself.

\section{Finiteness, degree and regularity}

Let us now focus on the degree of vertices and regularity. We make definitions with minimal
hypotheses as these are fundamental objects that should be added to Mathlib.

\subsection{Notions of finiteness}

Let $G$ be a graph.

\begin{definition}
  \label{def:locally_finite_at}
  Let $x$ be a vertex of $G$.
  We say $G$ is \emph{locally finite at $x$} if $\incset (x)$ is finite.
\end{definition}

\begin{definition}
  \label{def:locally_finite}
  \uses{def:locally_finite_at}
  $G$ is said to be \emph{locally finite} if, for all $x \in V(G)$, $G$ is locally finite at $x$.
\end{definition}

\begin{definition}
  \label{def:finite}
  We say $G$ is \emph{finite} if $V(G)$ and $E(G)$ are both finite.
\end{definition}

\begin{proposition}
  \label{def:finite_vs_locally_finite}
  \uses{def:finite,def:locally_finite}
  $G$ is finite if and only if $V(G)$ is finite and $G$ is locally finite.
\end{proposition}

\subsection{Degree and regularity}

Let $G$ be a graph.

\begin{definition}
  \label{def:degree}
  \uses{def:locally_finite_at}
  Let $x$ be a vertex of $G$. We assume $G$ is locally finite at $x$.
  The \emph{degree} of $x$ is the number of incident edges, counting loops twice:
  \begin{equation*}
    \mathrm{deg}(x) = \# \incset(x)
    + \# \loopset(x).
  \end{equation*}
\end{definition}

\begin{proof}
  The set of loops at $x$ is included in the incidence set of $x$ by IsLoopAt.inc. Hence, if $G$ is
  locally finite at $x$, then $x$ has a finite number of loops and the degree is well-defined.
\end{proof}

\begin{definition}
  \label{def:regular}
  \uses{def:locally_finite, def:degree}
  We assume that $G$ is locally finite.
  Let $d$ be a non-negative integer.
  We say $G$ is \emph{regular of degree $d$} if, for every $x \in V(G)$, $\deg (x) = d$.
\end{definition}

\section{$\C^*$-algebra and adjacency operator}

Let $G$ be a graph.
We will be interested in the Hilbert space $\ell^2(G) := \ell^2(V(G))$, which is defined as the set of
functions $f : V(G) \rightarrow \C$ which are square-summable, equipped with the inner product
$$\langle f_1, f_2 \rangle = \sum_{x \in V(G)} f_1(x) \overline{f_2(x)}.$$

The space $\B(\ell^2(G))$ of bounded linear operators on $\ell^2(G)$ is a unital $\C^*$-algebra.

\subsection{Constant fonctions}

\begin{definition}
  \label{def:constant_function}
  The constant function equal to $1$ is defined by
  \begin{equation}
    \const :
    \begin{cases}
      V(G) & \rightarrow \C \\
      x & \mapsto 1.
    \end{cases}
  \end{equation}
\end{definition}

\begin{lemma}
  \label{lem:constant_is_l2}
  \uses{def:constant_function}
  The constant function $\const$ belongs in $\ell^2(G)$
  if and only if $V(G)$ is finite.
\end{lemma}

\subsection{Adjacency operator}

Assume $G$ is locally finite.

\begin{definition}
  \label{def:adjacency_operator}
  \uses{def:locally_finite}
  We define the adjacency operator $a$ as the linear operator acting on the space of
  functions $f : V(G) \rightarrow \C$ by
  $$ a f (x) :=
  \sum_{e \in \incset(x)} f(\incother(e, x)) + \# \loopset(x) f(x).$$
\end{definition}

\begin{lemma}
  \label{lem:adjacency_of_constant}
  \uses{def:constant_function,def:adjacency_operator}
  For any $x \in V(G)$, $a \const(x) = \deg(x)$.
\end{lemma}

\begin{lemma}
  \label{lem:adjacency_regular}
  \uses{lem:adjacency_of_constant,def:regular}
  Let $d$ be a non-negative integer. Then, $G$ is regular of degree $d$ iff
  $a \const = d \const$.
\end{lemma}

We now enquire about the boundedness and norm of the adjacency operator.

\begin{lemma}
  \label{lem:adjacency_bounded}
  \uses{def:adjacency_operator,def:degree}
  We assume that there exists a non-negative integer $M$ such that:
  \begin{equation}
    \forall x \in V(G), \quad \deg(x) \leq M.
  \end{equation}
  Then the adjacency operator $a$ is bounded and $\| a \| \leq M$.
\end{lemma}

\begin{proposition}
  \label{prop:norm_adjacency_regular}
  \uses{lem:adjacency_regular,lem:adjacency_bounded,lem:constant_is_l2}
  We assume that $G$ is finite and $d$-regular. Then $\|a\|=d$.
\end{proposition}



% \section{Adjacency matrix}

% \begin{definition}
%   \label{def:n_labelled}
%   Let $n \geq 1$ be a natural integer. We say a graph $G = (V,E)$ is
%   \emph{$n$-labelled} if the vertex set $V$ is $\{1, \ldots, n\}$.
% \end{definition}

% \begin{definition}
%   \label{def:adjacency_matrix}
%   \uses{def:finite,def:n_labelled}
%   Let $G$ be a finite, $n$-labelled graph.
%   The \emph{adjacency matrix $A$} of $G$ is a $n \times n$ matrix with integer coefficients,
%   defined by
%   \begin{equation}
%     \forall (i, j) \in \{1, \ldots, n\}^2,
%     A_{i,j} :=
%     \begin{cases}
%       \# \{e \in E : e \text{ links } i \text{ and } j\} & \text{ if } i \neq j \\
%       2 \# \{e \in E : e \text{ is a loop at } i\} & \text{ if } i=j.
%     \end{cases}
%   \end{equation}
% \end{definition}

% \begin{lemma}
%   \label{lem:adjacency_symmetric}
%   \uses{def:adjacency_matrix}
%   The adjacency matrix of a graph is symmetric.
% \end{lemma}

% \begin{proof}
%   Direct from Definition \ref{def:adjacency_matrix} and \mathlibok{isLink symm}.
% \end{proof}

% In any space $\mathbb{R}^n$ for $n \geq 1$, we denote as $\mathbf{1}$ the constant vector with
% all coefficients equal to $1$.

% \begin{lemma}
%   \label{lem:adjacency_degree_vertex}
%   \uses{def:degree,def:adjacency_matrix}
%   Let $G$ be a finite, $n$-labelled graph, of adjacency matrix $A$.
%   For any $i \in V$, the degree of $G$ at $i$ is equal to $(A \mathbf{1})_i$.
% \end{lemma}

% \begin{proof}
%   This follows directly from the definition of degree and the adjacency matrix coefficients.
% \end{proof}

% \begin{corollary}
%   \label{cor:adjacency_regular}
%   \uses{def:regular,lem:adjacency_degree_vertex}
%   Let $G$ be a finite, $n$-labelled graph, of adjacency matrix $A$.
%   Let $d \geq 1$ be an integer.
%   The graph $G$ is regular of degree $d$ if and only if
%   $\mathbf{1}$ is an eigenvector of $A$ with eigenvalue $d$.
% \end{corollary}

% \begin{proof}
%   [todo]
% \end{proof}

% \begin{lemma}
%   \label{lemma:finite_of_label_and_reg}
%   \uses{def:regular,def:finite,def:n_labelled}
%   Let $G$ be a $n$-labelled $d$-regular graph. Then, $G$ is finite, $\# V(G) = n$
%   and $\#E(G) = nd/2$.
% \end{lemma}

% \begin{proposition}
%   \label{prop:spectral_radius}
%   \uses{def:adjacency_matrix,def:regular}
%   [to write]
% \end{proposition}

% \begin{definition}
%   \label{def:bipartite}
%   A graph $G$ is called \emph{bipartite} if there exists a partition $(V_1, V_2)$ of $V$ such that
%   every edge of $G$ links a vertex of $V_1$ to a vertex of $V_2$.
% \end{definition}

% \begin{proposition}
%   \label{prop:adjacency_bipartite}
%   \uses{def:bipartite,def:adjacency_matrix}
%   Let $G$ be a finite, $n$-labelled graph, of adjacency matrix $A$.
%   Let $d \geq 1$ be an integer.
%   We assume that the graph $G$ is regular of degree $d$.
%   Then, $G$ is bipartite if and only if $-d$ is an eigenvalue of $A$.
% \end{proposition}
