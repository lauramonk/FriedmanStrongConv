\chapter{Random regular graphs}

In this chapter, we introduce the definitions related to random regular graphs and their spectrum.
We refer the reader to \cite{chung1992} for a more detailed introduction and motivations.

\section{Definition of a graph}

In this section, we define a few simple notions of graph theory, to set conventions.

As in \mathlibok{Graph}, a \emph{graph} is a pair $G = (V, E)$ where $V$ is a set of vertices, and
$E$ a set of edges, and each edge links two unordered, possibly identical vertices.

Note that this definition allows for loops (i.e. edges going from one vertex to itself) and
multi-edges (i.e. multiple edges connecting two vertices). This convention is essential to the proof
we are aiming for, which is why we use \mathlibok{Graph} instead of the more developed
\mathlibok{SimpleGraph}.

Following \mathlibok{Inc}, we say an edge $e$ is \emph{incident} to a vertex $x$ if there exists a
vertex $y$ such that $e$ links $x$ and $y$. As \mathlibok{IsLoopAt}, an edge is a \emph{loop} based
at a vertex $x$ if it links the vertex $x$ to itself.

\section{Finiteness, degree and regularity}

Let us now focus on the degree of vertices and regularity. We make definitions with minimal
hypotheses as these are fundamental objects that should be added to Mathlib.

\begin{definition}
  \label{def:locally_finite_at}
  Let $G$ be a graph and $x$ be a vertex of $G$.
  We say $G$ is \emph{locally finite at $x$} if $x$ has a finite number of incident edges.
\end{definition}

\begin{definition}
  \label{def:degree}
  \uses{def:locally_finite_at}
  Let $G$ be a graph and $x$ be a vertex of $G$. We assume $G$ is locally finite at $x$.
  The \emph{degree} of $x$ is the number of incident edges, counting loops twice:
  \begin{equation*}
    \mathrm{deg}(x) = \# \{\text{edges incident to } x\}
    + \# \{ \text{loops based at } x\}.
  \end{equation*}
\end{definition}

\begin{proof}
  The set of loops at $x$ is included in the incidence set of $x$ by IsLoopAt.inc. Hence, if $G$ is
  locally finite at $x$, then $x$ has a finite number of loops and the degree is well-defined.
\end{proof}

\begin{definition}
  \label{def:locally_finite}
  \uses{def:locally_finite_at}
  A graph is said to be \emph{locally finite} if it is locally finite at every vertex.
\end{definition}

\begin{definition}
  \label{def:regular}
  \uses{def:locally_finite, def:degree}
  Let $d$ be a non-negative integer.
  Let $G$ be a locally finite graph.
  We say $G$ is \emph{regular of degree $d$} if every vertex of $G$ has degree $d$.
\end{definition}

We now conclude with the definition of a finite graph.

\begin{definition}
  \label{def:finite}
  \uses{def:locally_finite}
  A graph is \emph{finite} if its set of vertices is finite and it is locally finite.
\end{definition}

\section{Adjacency matrix}

\begin{definition}
  \label{def:n_labelled}
  Let $n \geq 1$ be a natural integer. We say a graph $G = (V,E)$ is
  \emph{$n$-labelled} if the vertex set $V$ is $\{1, \ldots, n\}$.
\end{definition}

\begin{definition}
  \label{def:adjacency_matrix}
  \uses{def:finite,def:n_labelled}
  Let $G$ be a finite, $n$-labelled graph.
  The \emph{adjacency matrix $A$} of $G$ is a $n \times n$ matrix with integer coefficients,
  defined by
  \begin{equation}
    \forall (i, j) \in \{1, \ldots, n\}^2,
    A_{i,j} :=
    \begin{cases}
      \# \{e \in E : e \text{ links } i \text{ and } j\} & \text{ if } i \neq j \\
      2 \# \{e \in E : e \text{ is a loop at } i\} & \text{ if } i=j.
    \end{cases}
  \end{equation}
\end{definition}

\begin{lemma}
  \label{lem:adjacency_symmetric}
  \uses{def:adjacency_matrix}
  Let $G$ be a finite, $n$-labelled graph.
  The adjacency matrix of $G$ is symmetric.
\end{lemma}

\begin{lemma}
  \label{lem:adjacency_degree_vertex}
  \uses{def:degree,def:adjacency_matrix}
  Let $G$ be a finite, $n$-labelled graph, of adjacency matrix $A$.
  For any $i \in V$, the degree of $G$ at $i$ is equal to $\sum_{j=1}^n A_{i,j}$.
\end{lemma}

\begin{proof}
  This follows directly from the definition of degree and the adjacency matrix coefficients.
\end{proof}

\begin{corollary}
  \label{cor:adjacency_regular}
  \uses{def:regular,lem:adjacency_degree_vertex}
  Let $G$ be a finite, $n$-labelled graph, of adjacency matrix $A$. Let $d \geq 1$ be an integer.
  The graph $G$ is regular of degree $d$ if and only if, for every $1 \leq i \leq n$,
  $\sum_{j=1}^n A_{i,j} = d$.
\end{corollary}

\begin{proof}
  This follows directly from the definition of regular graph \ref{def:regular} together with
  Lemma \ref{lem:adjacency_degree_vertex}.
\end{proof}
