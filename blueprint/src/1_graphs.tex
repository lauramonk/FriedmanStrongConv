\chapter{Spectral theory of regular graphs}

In this chapter, we introduce the definitions related to regular graphs and their spectrum.
We refer the reader to \cite{chung1992} for a more detailed introduction and motivations.

As in \mathlibok{Graph}, a \emph{graph} is a pair $G = (V, E)$ where $V$ is a set of vertices, and
$E$ a set of edges, and each edge links two unordered, possibly identical vertices.
The notations $V(G)$ and $E(G)$ respectively denote the subsets of $V$ and $E$ corresponding to
vertices or edges of $G$.

Note that this definition allows for loops (i.e. edges going from one vertex to itself) and
multi-edges (i.e. multiple edges connecting two vertices). This convention is essential to the proof
we are aiming for, which is why we use \mathlibok{Graph} instead of the more developed
\mathlibok{SimpleGraph}.

Following \mathlibok{Inc}, we say an edge $e$ is \emph{incident} to a vertex $x$ if there exists a
vertex $y$ such that $e$ links $x$ and $y$. As \mathlibok{IsLoopAt}, an edge is a \emph{loop} based
at a vertex $x$ if it links the vertex $x$ to itself.

\section{Finiteness, degree and regularity}

Let us now focus on the degree of vertices and regularity. We make definitions with minimal
hypotheses as these are fundamental objects that should be added to Mathlib.

\begin{definition}
\label{def:empty}
We say a graph $G$ is empty if $V(G) = \emptyset$.
\end{definition}

\subsection{Notions of finiteness}

Let $G$ be a graph.

\begin{definition}
  \label{def:locally_finite_at}
  Let $x$ be a vertex of $G$.
  We say $G$ is \emph{locally finite at $x$} if $\incset (x)$ is finite.
\end{definition}

\begin{definition}
  \label{def:locally_finite}
  \uses{def:locally_finite_at}
  $G$ is said to be \emph{locally finite} if, for all $x \in V(G)$, $G$ is locally finite at $x$.
\end{definition}

\begin{definition}
  \label{def:finite}
  We say $G$ is \emph{finite} if $V(G)$ and $E(G)$ are both finite.
\end{definition}

\begin{proposition}
  \label{def:finite_vs_locally_finite}
  \uses{def:finite,def:locally_finite}
  $G$ is finite if and only if $V(G)$ is finite and $G$ is locally finite.
\end{proposition}

\subsection{Degree and regularity}

Let $G$ be a graph.

\begin{definition}
  \label{def:degree}
  \uses{def:locally_finite_at}
  Let $x$ be a vertex of $G$. We assume $G$ is locally finite at $x$.
  The \emph{degree} of $x$ is the number of incident edges, counting loops twice:
  \begin{equation*}
    \mathrm{deg}(x) = \sum_{e \in \incset(x)}
    2^{\isloop(e)}.
  \end{equation*}
\end{definition}

\begin{definition}
  \label{def:regular}
  \uses{def:locally_finite, def:degree}
  Let $d$ be a non-negative integer.
  We say $G$ is \emph{regular of degree $d$} if it is not empty, locally finite and,
  for every $x \in V(G)$, $\deg (x) = d$.
\end{definition}

\begin{definition}
  \label{def:locally_bounded}
  \uses{def:degree,def:locally_finite}
  We say $G$ is \emph{locally bounded} if
  it is locally finite and there exists a constant $D$ such that,
  for all $x \in V(G)$, $\deg(x) \leq D$.
\end{definition}

\section{Adjacency operator}

Let $G$ be a graph.
We will be interested in the Hilbert space $\ell^2(G) := \ell^2(V(G))$, which is defined as the set of
functions $f : V(G) \rightarrow \C$ which are square-summable, equipped with the inner product
$$\langle f_1, f_2 \rangle = \sum_{x \in V(G)} f_1(x) \overline{f_2(x)}.$$

The space $\B(\ell^2(G))$ of bounded linear operators on $\ell^2(G)$ is a unital $\C^*$-algebra.
Note that, in this context, $0$ is the constant function equal to $0$.

\subsection{Constant fonctions}

\begin{definition}
  \label{def:constant_function}
  The constant function equal to $1$ is defined by
  \begin{equation}
    \const :
    \begin{cases}
      V(G) & \rightarrow \C \\
      x & \mapsto 1.
    \end{cases}
  \end{equation}
\end{definition}

\begin{lemma}
  \label{lem:constant_neq_zero}
  \uses{def:empty,def:constant_function}
  Assume $G$ is not empty. Then the constant function $\const$ is not equal to $0$.
\end{lemma}

\begin{proof}
  If $G$ is not empty, then by Definition \ref{def:empty}, there exists $x \in V(G)$.
  Then $\const (x) = 1 \neq 0$, and hence $\const$ is not equal to the constant function $0$.
\end{proof}

\begin{lemma}
  \label{lem:constant_is_l2}
  \uses{def:constant_function}
  The constant function $\const$ belongs in $\ell^2(G)$
  if and only if $V(G)$ is finite.
\end{lemma}

\subsection{Definition of the adjacency operator and elementary properties}

Assume $G$ is locally finite.

\begin{definition}
  \label{def:adjacency_operator}
  \uses{def:locally_finite}
  We define the adjacency operator $a$ as the linear operator acting on the space of
  functions $f : V(G) \rightarrow \C$ by
  $$ a f (x) :=
  \sum_{e \in \incset(x)}
  2^{\isloop(e)} f(\incother(e,x)).$$
\end{definition}

\begin{lemma}
  \label{lem:adjacency_of_constant}
  \uses{def:constant_function,def:adjacency_operator}
  For any $x \in V(G)$, $a \const(x) = \deg(x)$.
\end{lemma}

\begin{proof}
  Let $x \in V(G)$. We have
  \begin{align*}
    a \const(x)
    & = \sum_{e \in \incset(x)}
    2^{\isloop(e)}\const(\incother(e,x))
    & \text{by Definition \ref{def:adjacency_operator}} \\
    & = \sum_{e \in \incset(x)}
    2^{\isloop(e)}
    & \text{by Definition \ref{def:constant_function}} \\
    & = \deg(x)
    & \text{by Definition \ref{def:degree}}.
  \end{align*}
\end{proof}

\subsection{Norm of the adjacency operator}

Let $G$ be a locally finite graph.
The aim of this subsection is to prove the following.

\begin{proposition}
  \label{prop:norm_adjacency_leq_sup_degree}
  \uses{lem:bound_a_sup_degree}
  Assume $G$ is locally bounded.
  Then the adjacency operator is a bounded operator on $\ell^2(G)$ and
  its norm satisfies $\|a\| \leq \sup_{x \in V(G)} \deg(x)$.
\end{proposition}

Here is a key lemma.

\begin{lemma}
  \label{lem:bound_a_degree_x}
  \uses{def:adjacency_operator,def:degree}
  For any function $f : V(G) \rightarrow \C$, any vertex $x \in V(G)$,
  \begin{equation}
    |af(x)|^2
    \leq \deg(x)
    \sum_{e \in \incset(x)} 2^{\isloop(e)} |f(\incother(e,x))|^2.
  \end{equation}
\end{lemma}

\begin{proof}
  Let $f : V(G) \rightarrow \C$ and $x \in V(G)$.
  By Definition \ref{def:adjacency_operator},
  \begin{equation}
    |af(x)|^2
    = \left| \sum_{e \in \incset(x)} 2^{\isloop(e)} f(\incother(e, x))\right|^2.
  \end{equation}
  By the \mathlibok{Cauchy-Schwarz} inequality,
  \begin{equation*}
    |af(x)|^2
    \leq
    \left(\sum_{e \in \incset(x)} 2^{\isloop(e)}
    \right)
    \left( \sum_{e \in \incset(x)} 2^{\isloop(e)} |f(\incother(e, x))|^2
    \right).
  \end{equation*}
  This leads to the claim by Definition \ref{def:degree}.
\end{proof}

Then Proposition \ref{prop:norm_adjacency_leq_sup_degree} is a trivial consequence of the
following lemma, together with the definition of bounded operator and operator norm.

\begin{lemma}
  \label{lem:bound_a_sup_degree}
  \uses{lem:bound_a_degree_x,def:locally_bounded}
  Assume $G$ is locally bounded. Then, for any $f \in \ell^2(G)$,
  $a f \in \ell^2(G)$ and
  \begin{equation}
    \|a f\| \leq \sup_{x \in V(G)} \deg(x) \times \|f\|.
  \end{equation}
\end{lemma}

\begin{proof}
  To write, basically we need to sum Lemma \ref{lem:bound_a_degree_x} for $x \in V(G)$
  (find the correct summation result), change the order of summation (same)
  and observe that
  \begin{align*}
    \sum_{x \in V(G)}
    \sum_{e \in \incset(x)}
    2^{\isloop(e)}
    |f(\incother(e,x))|^2
    & = \sum_{x \in V(G)} \deg(x) |f(x)|^2 \\
    & \leq \sup_{x \in V(G)} \deg(x) \times \|f\|^2.
  \end{align*}
\end{proof}

\subsection{The case of regular graphs}

Let $d$ be a non-negative integer.

\begin{lemma}
  \label{lem:adjacency_of_constant_regular}
  \uses{lem:adjacency_of_constant,def:regular}
  Assume $G$ is not empty and locally finite. Then
  $G$ is regular of degree $d$ iff $a \const = d \const$.
\end{lemma}

\begin{proof}
  By Definition \ref{def:regular},
  $G$ is regular of degree $d$ if and only if for all $x \in V(G)$, $\deg(x)=d$.
  By Lemma \ref{lem:adjacency_of_constant}, this is equivalent to
  having, for all $x \in V(G)$, $a \const (x) = d$, which is equivalent to
  $a \const = d \const$ by Definition \ref{def:constant_function}.
\end{proof}

\begin{lemma}
  \label{lem:constant_eigenfunction}
  \uses{lem:constant_is_l2,lem:constant_neq_zero,lem:adjacency_of_constant_regular}
  If $G$ is finite and regular of degree $d$,
  then $d$ is an eigenvalue of the adjacency
  operator with associated eigenfunction the constant function $\const$.
\end{lemma}

\begin{proof}
  Since $G$ is finite, $\const \in \ell^2(G)$ by Lemma \ref{lem:constant_is_l2}.
  Furthermore, $\const \neq 0$ by Lemma \ref{lem:constant_neq_zero}
  and since $G$ is not empty by Definition \ref{def:regular}.
  By Lemma \ref{lem:adjacency_of_constant_regular}, if $G$ is regular of degree $d$,
  then $a \const = d \const$.  We conclude by the definition of eigenvalue and eigenvector.
\end{proof}

\begin{proposition}
  \label{prop:norm_adjacency_regular}
  \uses{prop:norm_adjacency_sup_degree,lem:constant_eigenfunction}
  If $G$ is finite and regular of degree $d$, then $\|a\|=d$.
\end{proposition}

\begin{proof}
  Since $G$ is regular of degree $d$, for all $x \in V(G)$, $\deg(x) \leq d$. Then, by
  Lemma \ref{prop:norm_adjacency_sup_degree}, $a$ is bounded and $\|a \| \leq d$.
  By Lemma \ref{lem:constant_eigenfunction}, $d$ is an eigenvalue of $a$ and hence $\|a\| \geq d$,
  which is enough to conclude.
\end{proof}

% \section{Adjacency matrix}

% \begin{definition}
%   \label{def:n_labelled}
%   Let $n \geq 1$ be a natural integer. We say a graph $G = (V,E)$ is
%   \emph{$n$-labelled} if the vertex set $V$ is $\{1, \ldots, n\}$.
% \end{definition}

% \begin{definition}
%   \label{def:adjacency_matrix}
%   \uses{def:finite,def:n_labelled}
%   Let $G$ be a finite, $n$-labelled graph.
%   The \emph{adjacency matrix $A$} of $G$ is a $n \times n$ matrix with integer coefficients,
%   defined by
%   \begin{equation}
%     \forall (i, j) \in \{1, \ldots, n\}^2,
%     A_{i,j} :=
%     \begin{cases}
%       \# \{e \in E : e \text{ links } i \text{ and } j\} & \text{ if } i \neq j \\
%       2 \# \{e \in E : e \text{ is a loop at } i\} & \text{ if } i=j.
%     \end{cases}
%   \end{equation}
% \end{definition}

% \begin{lemma}
%   \label{lem:adjacency_symmetric}
%   \uses{def:adjacency_matrix}
%   The adjacency matrix of a graph is symmetric.
% \end{lemma}

% \begin{proof}
%   Direct from Definition \ref{def:adjacency_matrix} and \mathlibok{isLink symm}.
% \end{proof}

% In any space $\mathbb{R}^n$ for $n \geq 1$, we denote as $\mathbf{1}$ the constant vector with
% all coefficients equal to $1$.

% \begin{lemma}
%   \label{lem:adjacency_degree_vertex}
%   \uses{def:degree,def:adjacency_matrix}
%   Let $G$ be a finite, $n$-labelled graph, of adjacency matrix $A$.
%   For any $i \in V$, the degree of $G$ at $i$ is equal to $(A \mathbf{1})_i$.
% \end{lemma}

% \begin{proof}
%   This follows directly from the definition of degree and the adjacency matrix coefficients.
% \end{proof}

% \begin{corollary}
%   \label{cor:adjacency_regular}
%   \uses{def:regular,lem:adjacency_degree_vertex}
%   Let $G$ be a finite, $n$-labelled graph, of adjacency matrix $A$.
%   Let $d \geq 1$ be an integer.
%   The graph $G$ is regular of degree $d$ if and only if
%   $\mathbf{1}$ is an eigenvector of $A$ with eigenvalue $d$.
% \end{corollary}

% \begin{proof}
%   [todo]
% \end{proof}

% \begin{lemma}
%   \label{lemma:finite_of_label_and_reg}
%   \uses{def:regular,def:finite,def:n_labelled}
%   Let $G$ be a $n$-labelled $d$-regular graph. Then, $G$ is finite, $\# V(G) = n$
%   and $\#E(G) = nd/2$.
% \end{lemma}

% \begin{proposition}
%   \label{prop:spectral_radius}
%   \uses{def:adjacency_matrix,def:regular}
%   [to write]
% \end{proposition}

% \begin{definition}
%   \label{def:bipartite}
%   A graph $G$ is called \emph{bipartite} if there exists a partition $(V_1, V_2)$ of $V$ such that
%   every edge of $G$ links a vertex of $V_1$ to a vertex of $V_2$.
% \end{definition}

% \begin{proposition}
%   \label{prop:adjacency_bipartite}
%   \uses{def:bipartite,def:adjacency_matrix}
%   Let $G$ be a finite, $n$-labelled graph, of adjacency matrix $A$.
%   Let $d \geq 1$ be an integer.
%   We assume that the graph $G$ is regular of degree $d$.
%   Then, $G$ is bipartite if and only if $-d$ is an eigenvalue of $A$.
% \end{proposition}
